Este capítulo demonstra PF no processamento de listas.

\todo[noline]{“Dimensions of evaluation” em inglês, talvez “dimensões de análise” ou “de verificação” seja melhor?}
\section{Dimensões de Avaliação}
\label{sec:org049f7a0}

As dimensões utilizadas neste trabalho são baseadas nas DCs. Assim usamos
critérios estabelecidos, o que facilita a comparação de nossos resultados com
trabalhos baseados nas DCs. O \emph{framework} completo possui 14 dimensões, mas os
autores recomendam usar um subconjunto apropriado para o artefato em mãos.

\begin{description}
\item[{Nível de abstração}] \emph{Disponibilidade e tipos de mecanismos de abstração}

O sistema fornece alguma forma de definir novos termos com a notação pra
que eles possam ser extendidos afim de descrever ideias claramente? Os
detalhes podem ser encapsulados? O sistema insiste em definir novos
termos? Qual o número de novos conceitos de alto nível devem ser
aprendidos pra se fazer uso do sistema? Eles são fáceis de aprender e
usar?

Cada novo conceito é um empecilho pra aprendizagem e aceitação, mas
também pode tornar um código complexo mais compreensível.

\item[{Proximidade de descrição}] \todo{Traduzido de “Closeness of mapping”, poderia ser “Proximidade de mapeamento”} \emph{Semelhança entre representação e o domínio}

O quão relacionado é a notação com o resultado que ela descreve, ou
melhor, o domínio do problema? Que partes parecem ser uma forma
particularmente estranha de descrever algo?

\item[{Dependências ocultas}] \emph{Vínculos importantes implícitos entre entidades}

As dependências entre as entidades da notação são visíveis ou ocultas?
Todas as dependências são especificadas em ambas as direções? Alterações
locais podem ter efeitos globais confusos?

Se uma entidade cita outra, que por sua vez cita uma terceira, a
alteração do valor da terceira entidade pode desencadear efeitos
inesperados. O problema chave não é o fato de A depender de B, mas sim
que a dependência não é visível. Um caso bem conhecido de dependência
oculta é o “problema da classe base frágil”\footnote{Veja: \url{https://en.wikipedia.org/wiki/Fragile\_base\_class}.}.

\item[{Propensão a erros}] \emph{Notação incita erros}

Até que ponto a notação influência o programador a cometer um erro? Fazer
algumas coisas parece ser particularmente complexo ou difícil, p. ex.,
juntar várias coisas?

Em muitas linguagens dinâmicas com definições implícitas de
variáveis\footnote{Isto é, quando não se precede uma definição de variável
com \texttt{var} ou \texttt{let} por exemplo.}, um erro de tipagem em uma variável pode
de repente levar a erros difíceis de encontrar já que o IDE nem sempre
pode apontar tal erro devido a dinamicidade na linguagem. A inicialização
implícita de variáveis com valor \texttt{null} pode levar a uma exceção de
ponteiro nulo se o programador esquecer de inicializar as variáveis
corretamente antes de usá-las.

\item[{Difusão        }] \todo{Diffuseness/terseness: difusão/concisão, ou dispersão} \emph{Verbosidade da linguagem}

Quantos símbolos ou quanto espaço a notação requer pra produzir um certo
resultado ou expressar uma ideia. Que tipos de coisas ocupam mais espaço
para se descrever?

\item[{Viscosidade    }] \emph{Resistência a mudanças}

Existe alguma barreira contra mudança na notação? Quanto esforço é
necessário pra fazer uma alteração num programa expresso na notação?

Num sistema viscoso o usuário precisa realizar várias passos para
concluir uma tarefa. Alterar o tipo de returno de uma função pode causar
erros em várias partes do código onde a função é chamada. Nesses casos um
IDE pode ajudar muito.
\end{description}


Outras dimensões que ficaram de fora são:

\begin{itemize}
\item Comprometimento prematuro: \emph{limitações na ordem de fazer as coisas}
\item Expressividade: \emph{o propósito de uma entidade pode ser rapidamente determinado}
\item Consistência: \emph{semânticas parecidas são apresentadas em estilo sintático
parecido}
\item Operações mentais difíceis: \emph{alto esforço cognitivo pra realizar tarefas}
\item Notação secundária: \emph{informações extras por meio de notações além das
formais}
\end{itemize}


Em outras situações, e dependendo dos artefatos de informação, essas dimensões
poderiam ser úteis. Se tivéssemos focado mais nas ferramentas e no processo de
criação dos estudos de caso então as seguintes dimensões poderiam ser usadas:

\begin{itemize}
\item Visibilidade: \emph{habilidade de ver os componentes claramente}
\item Análise progressiva: \emph{trabalho realizado pode ser verificado em qualquer
momento}
\item Provisoriedade\todo{'momentaneidade' ou 'transitoriedade'}: \emph{grau
de comprometimento com uma ação ou marco}
\end{itemize}


\section{Processamento de Listas}
\label{sec:org171044f}
\subsection{Projeção de valores com \texttt{map()}}
\label{sec:orgb51c5c5}
\begin{itemize}
\item Adicionar \(1\) a cada número de uma lista
\item Multiplicar os números de uma lista
\end{itemize}

\subsection{Seleção de valores com \texttt{filter()}}
\label{sec:orgae38ddd}
Remover nomes que não começam com ‘S’

\subsection{Nivelamento de valores com \texttt{concatAll()}}
\label{sec:org35a5bbe}
\subsection{Redução de valores com \texttt{reduce()}}
\label{sec:orgcf561ff}
\subsection{Agrupamento de valores com \texttt{zip()}}
\label{sec:org7150be0}

\section{Coordenação de Eventos}
\label{sec:org390307a}
\subsection{Contador}
\label{sec:org6426061}
\subsection{Reserva de voo}
\label{sec:orga5e552b}
