Um programa de computador é expresso por um conjunto de sentenças em alguma
linguagem de programação.\footnote{Uma sentença (\emph{statement}) pode conter uma ou várias expressões ou instruções.
Uma única instrução numa linguagem de alto nível pode representar múltiplas
instruções de máquinas.
Programas consistem de instruções e expressões.
Uma expressão é um grupo de símbolos que representa um valor.}
A forma como as sentenças de uma linguagem são executadas pelo computador é
definida por um \emph{modelo de computação} \cite{roy2004}.

O código fonte de um programa consiste de uma série de instruções que expressam
uma sequência de comandos a se seguir durante a execução — esse período é
conhecido como \emph{tempo de execução}, do inglês \emph{runtime}.
Tradicionalmente, as sentenças de uma linguagem de programação denotam essa
sequência de forma explícita.
Tais linguagens são baseadas no modelo \emph{imperativo} de computação, e são
denominadas \emph{linguagens imperativas}.

\section{Linguagens de Programação}
\label{sec:org7b5e110}
\label{sec:langs}

Linguagens de programação são mais simples que linguagens naturais, no
entanto, elas ainda podem conter uma sintaxe surpreendentemente rica, um
conjunto de abstrações, e bibliotecas auxiliares.
Esse é essencialmente o caso de linguagens usadas para resolver problemas
reais do dia-a-dia.
\textcite{roy2004} as chamam de linguagens \emph{práticas}, que são “\textelp{}
como a caixa de ferramentas de um mecânico experiente: há várias ferramentas
diferentes para finalidades diferentes e todas estão lá por uma razão.” (p.
33; tradução nossa).

Todas as linguagens de programação possuem elementos primitivos para a
descrição de dados e das transformações, ou processos, aplicados à eles —
como a adição de dois números ou a seleção de um item de uma coleção.
Essas primitivas são definidas por regras de sintaxe — a gramática — e pela
semântica — o significado.

Linguagens imperativas geralmente oferecem comandos para lidar com estado em
tempo de execução, como declaração e atribuição de variáveis, e comandos
para controlar o caminho que o programa deve seguir, como os que decidem a
ordem de execução das sentenças — na literatura essa ordem é chamada de
\emph{fluxo de controle} de um programa.

Durante sua execução o programa segue um caminho de acordo com seu \emph{estado}
interno — ou \emph{memória}, o que um programa se lembra enquanto está ‘rodando’.
Programas com estado interno, ou \emph{statefull} em inglês, são projetados para
lembrar de eventos anteriores ou de interações com o usuário.
A informação recordada é denominada o estado do programa \cite{rouse2005}.
